%%%%%%%%%%%%%%%%%%%%%%%%%%%%%%%%%%%%%%%%%
% University Assignment Title Page 
% LaTeX Template
% Version 1.0 (27/12/12)
%
% This template has been downloaded from:
% http://www.LaTeXTemplates.com
%
% Original author:
% WikiBooks (http://en.wikibooks.org/wiki/LaTeX/Title_Creation)
%
% License:
% CC BY-NC-SA 3.0 (http://creativecommons.org/licenses/by-nc-sa/3.0/)
% 
% Instructions for using this template:
% This title page is capable of being compiled as is. This is not useful for 
% including it in another document. To do this, you have two options: 
%
% 1) Copy/paste everything between \begin{document} and \end{document} 
% starting at \begin{titlepage} and paste this into another LaTeX file where you 
% want your title page.
% OR
% 2) Remove everything outside the \begin{titlepage} and \end{titlepage} and 
% move this file to the same directory as the LaTeX file you wish to add it to. 
% Then add \input{./title_page_1.tex} to your LaTeX file where you want your
% title page.
%
%%%%%%%%%%%%%%%%%%%%%%%%%%%%%%%%%%%%%%%%%
%\title{Title page with logo}
%----------------------------------------------------------------------------------------
%	PACKAGES AND OTHER DOCUMENT CONFIGURATIONS
%----------------------------------------------------------------------------------------

\documentclass[12pt]{article}
\usepackage[english]{babel}
\usepackage[utf8x]{inputenc}
\usepackage{amsmath}
\usepackage{graphicx}
\usepackage[colorinlistoftodos]{todonotes}
\usepackage{subcaption}

\begin{document}

\begin{titlepage}

\newcommand{\HRule}{\rule{\linewidth}{0.5mm}} % Defines a new command for the horizontal lines, change thickness here

\center % Center everything on the page
 
%----------------------------------------------------------------------------------------
%	HEADING SECTIONS
%----------------------------------------------------------------------------------------

% Name of your university/college
\textsc{\LARGE Instituto Superior T\'{e}cnico}\\[1.5cm]
% Major heading such as course name
\textsc{\Large ISR}\\[0.5cm]
% First Minor heading such as course title
\textsc{\large Report}\\[0.25cm]
% Second Minor heading such as course title
\textsc{\small State Of The Art Milestone}\\[0.25cm]

%----------------------------------------------------------------------------------------
%	TITLE SECTION
%----------------------------------------------------------------------------------------

\HRule \\[0.5cm]
{ \large \bfseries State Of The Art Essay: A First Approach}\\[0.25cm] % Title of your document
\HRule \\[0.5cm]
 
%----------------------------------------------------------------------------------------
%	AUTHOR SECTION
%----------------------------------------------------------------------------------------

\begin{minipage}{0.4\textwidth}
\begin{flushleft} \large
\emph{Author:}\\
Francisco Maria \textsc{Calisto} % Your name
\end{flushleft}
\end{minipage}
~
\begin{minipage}{0.4\textwidth}
\begin{flushright} \large
\emph{Coordinator:} \\
Professor Jacinto \textsc{Peixoto} % Coordinator's Name
\end{flushright}
~
\begin{flushright} \large
\emph{Co-Coordinator:} \\
Professor Daniel \textsc{Gon\c{c}alves} % Co-Coordinator's Name
\end{flushright}
\end{minipage}\\[2cm]

% If you don't want a supervisor, uncomment the two lines below and remove the section above
%\Large \emph{Author:}\\
%John \textsc{Smith}\\[3cm] % Your name

%----------------------------------------------------------------------------------------
%	DATE SECTION
%----------------------------------------------------------------------------------------

{\large 07/03/2016}\\[1cm] % Date, change the \today to a set date if you want to be precise

%----------------------------------------------------------------------------------------
%	LOGO SECTION
%----------------------------------------------------------------------------------------

\includegraphics{ist-logo.png}\\[0.5cm] % Include a department/university logo - this will require the graphicx package

\includegraphics{isr-logo.png}\\[0.5cm] % Include a department/university logo - this will require the graphicx package
 
%----------------------------------------------------------------------------------------

\vfill % Fill the rest of the page with whitespace

\end{titlepage}

\section{Introduction}

The Medical Imaging Multimodality Breast Cancer Diagnosis is a topic of great interest, it has been the subject of much work in the world of medicine, but few developments in terms of innovation in the computational world. An Application like this has a wide spectrum fields reference from surveillance based systems to medical application.

A vast majority of masses and calcifications can be accurately diagnosed from cytological features [1] of the cells that constitute them. However, the diagnostic accuracy depends on the training, experience, and many indefinite factors of interpretation of the medical expert in cytological evaluation.

There was, in fact, some developments in the past facing the classification system Computer-Based [2, 3] that assists in the diagnosis of breast cells based on visual assessment of characteristics of the cells. [4] A set of cytologic features, previously evaluated visually, are now replaced by digital ones, evaluated by image analysis. In this project we will compare the human precision in cytological diagnosis of breast cancer by digital image analysis accuracy combined with Computer-Based Machine Learn classification.

\section{Overview}

These systems are typically single-user oriented that is, designed to support individual tasks such as notations and information visualisation. This personal and task-oriented approach for clinical software provides little support for the aggregation of resources and tools required in carrying out higher-level activities for multimodality of medical imaging. It is left to the user to aggregate such resources and tools in meaningful bundles according to the activity at hand, and users often have to reconfigure this aggregation manually when shifting between a set of parallel activities and machines.

A propitious number of studies have shown that clinical professionals in the act of organising and thinking in their work routines, that are carried out search of general objectives, often in collaboration with others [9, 10, 11], are significant mental and manual overhead associated with handling of parallel work and interruptions [5, 8], and user interfaces in the current operating systems fail to provide adequate support in the resumption of the previous activities and for an easy switching between parallel activities [6, 7].

\section{User Interface Contextualisation}

The first step in successfully analysing the digital image is to specify the exact location of each masses nucleus or calcifications nucleus. The image is projected onto a computer screen, and the clinical medical operator uses, preferentially, a mouse button that will trace a rough outline of each visible masses (Figure 1) nucleus. On the other hand, the clinical medical operator will mark with dots the calcification (Figure 2) nucleus of cells.

% Commands to include a figure:
\begin{figure}[!hbt]
\centering
\includegraphics[width=0.75\textwidth]{masses.png}
\caption{\label{fig:frog}Mammographic image of a high-density mass (arrow)
}
\end{figure}

% Commands to include a figure:
\begin{figure}[!hbt]
\centering
\includegraphics[width=0.75\textwidth]{calcifications.png}
\caption{\label{fig:frog}Mammogram – shows calcifications, an early sign of breast cancer
}
\end{figure}

\clearpage

\section{Introductory Related Work}

Other systems have been designed to provide more direct support for managing multiple concurrent activities associated with a large amount of digital material and tools. In our discussion below, we will follow clinical imaging tools and non-clinical imaging tools that propose for a multimodal and non-multimodal views.
For this paper report we approach many others like an Activity-Based Computing (ABC) for Medical Work in Hospitals [12] that presents the concept, which seeks to create computational support for human activities contributing to the growing research on support for human activities, mobility, collaboration, and context-aware computing. To sum up, activity-based computing builds and expands upon prior work within each of the areas described as activity management, virtual window management, collaboration support systems and context-awareness. However the last topics, it does not approach a multimodal view of images, but it was great on other fields of understanding the context and most of the problem/solutions.
By using computer based image analysing we research a Breast Cytology Diagnosis Via Digital Image Analysis [1] paper work that brings us an improvement on the diagnostic accuracy of breast fine needle aspirate (FNA) goal where an interactive computer system has been developed for evaluating cytologic features derived directly from a digital scan of breast FNA slides. The system uses computer vision technology techniques to analyse cell nuclei and classifies them using an inductive method based on linear programming.

\clearpage

\begin{thebibliography}{}
\bibitem{} William H. Wolberg, W. Nick Street, Olvi L. Mangasarian. 1993. Breast Cytology Diagnosis. \emph{Via Digital Image Analysis}.
\bibitem{} Bennett KP, Mangasarian OL. Robust linear programming discrimination of two linearly inseparable sets. \emph{Optimization Methods and Software 1:23-34}, 1992.
\bibitem{} Mangasarian OL. Multi-surface method of pattern separation. \emph{IEEE Transactions on Information Theory IT-14:801-807}, 1968.
\bibitem{} Wolberg WH, Mangasarian OL. 1992. Multisurface method of pattern separation for medical diagnosis applied to breast cytology. \emph{Proceedings of the National Academy of Sciences}. U.S.A. 87:9193-9196.
\bibitem{} CZERWINSKI, M., HORVITZ, E., AND WILHITE, S. A diary study of task switching and interruptions. \emph{Proceedings of the Conference on Human Factors in Computing Systems}. ACM Press, 175–182.
\bibitem{} ROBERTSON, G., HORVITZ, E., CZERWINSKI, M., BAUDISCH, P., HUTCHINGS, D. R., MEYERS, B., ROBBINS, D., AND SMITH, G. Scalable fabric: Flexible task management. \emph{Proceedings of the Working Conference on Advanced Visual Interfaces}. ACM Press, 85–89.
\bibitem{} ROBERTSON, G., VAN DANTZICH, M., ROBBINS, D., CZERWINSKI, M., HINCKLEY, K., RISDEN, K., THIEL, D., AND GOROKHOVSKY, V. 2000. The task gallery: a 3d window manager. \emph{Proceedings of the SIGCHI Conference on Human Factors in Computing Systems}. ACM Press, 494–501.
\bibitem{} SMITH, G., BAUDISCH, P., ROBERTSON, G. G., CZERWINSKI, M., MEYERS, B., ROBBINS, D., AND ANDREWS, D. 2003. Groupbar: The taskbar evolved. \emph{Proceedings of the Australian Computer-Human Interaction Special Interest Group (OZCHI)}.
\bibitem{} BARDRAM, J. E. AND CHRISTENSEN, H. B. 2004. Real-time collaboration in activity-based architectures. \emph{Proceedings of 4th Working IEEE/IFIP Conference on Software Architecture (WICSA ’04)}. IEEE Press, 325–329.
\bibitem{} GONZALEZ, V. M. AND MARK, G. 2004. “Constant, constant, multi-tasking craziness”: Managing multiple working spheres. \emph{Proceedings of the SIGCHI Conference on Human Factors in Computing Systems (CHI ’04)}. ACM Press, 113–120.
\bibitem{} MARK, G., GONZALEZ, V. M., AND HARRIS, J. 2005. No task left behind?: Examining the nature of fragmented work. \emph{Proceedings of the SIGCHI Conference on Human Factors in Computing Systems (CHI ’05)}. ACM Press, 321–330.
\bibitem{} BARDRAM, J. E. 2009. Activity-Based Computing for Medical Work in Hospitals. \emph{ACM Transactions on Computer-Human Interaction}.
\end{thebibliography}

\end{document}
