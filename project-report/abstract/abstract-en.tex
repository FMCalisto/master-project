%!TEX root = ../dissertation.tex

\begin{otherlanguage}{english}
\begin{abstract}
% Set the page style to show the page number
\thispagestyle{plain}
\abstractEnglishPageNumber

Breast cancer is one of the most commonly occurring type of cancer among women \cite{}, the main strategy to reduce morbidity and also mortality being early detection and treatment based on medical imaging technologies. The current workflow applied in breast cancer diagnosis involves several imaging multi-modalities. A need for multi-modal imaging in breast cancer diagnosis is based on the fact that no single modality has the specificity and the sensitivity high enough for reliable diagnosis \cite{}. Nevertheless, their combination can significantly increase diagnostic accuracy \cite{}, this also reduces the number of unnecessary biopsies, which leads to better patient care and lower health care costs.

It is becoming increasingly apparent in medical image analysis that multiple imaging multi-modalities are required for the accurate treatment and diagnosis of the disease. An example to this, a patient will initially undertake a \gls{MG} for breast cancer diagnosis, with abnormal cases being further investigated using a combination of tomo-synthesis, in our case of study, \gls{US} and \gls{MRI}. This images then guide the physicians final diagnosis of suspicious lesions in order to achieve acceptable levels of specificity.

Our work is to develop techniques that enable the development of an improved user interface breast cancer diagnosis multimodality of image system based on any combination of \gls{MG}, \gls{US}, \gls{MRI} and Text Data. The plan involves the development and design of an user interface for automatic detection, segmentation and classification from breast \gls{MG}, \gls{US} and \gls{MRI}, as well as, textual data notations and information visualisation.

% Keywords
\begin{flushleft}

\keywords{medical, imaging, multimodality, breast cancer, diagnosis, user interface}

\end{flushleft}

\end{abstract}
\end{otherlanguage}
