%TEX root = ../dissertation.tex

\chapter{Goals}
\label{chapter:goals}

Development of the User Interface for Diagnosis of Breast Cancer in Medical Imaging Multimodality.

It is intended to develop an User Interface for monitoring and diagnosis of breast lesions in various medical imaging modalities. Imaging modalities to include in the work are:

- Mammography (including the views caudal-skull and oblique);

- Ultrasound;

- MRI volumes;

With protocol already signed with the Fernando Fonseca Hospital, it is intended that this interface has two features:

(i) - Build a database with annotations in multimodality mammography image.

Provide the user (doctor) facility to draw / write down masses and calcifications, as well as the corresponding BI-RADS for each imaging modality. This annotation process can be built during the examination, and thus it is possible to build a database of medical notes.

(ii) - Follow-up of the patient. With this feature is to allow the doctor automate a multimodality inspection for the patient.

Based on the patient's identification (eg, via a query on the ID), and for a given type of mammography imaging, the system must return all images of this patient over a period of time (eg. Two or more years) entered by the doctor, and show these images (pre-recorded). This feature is critical for diagnosis because it allows, through information visualisation, observing not only the calcifications density and the morphological evolution of the masses in that time period.