%!TEX root = ../dissertation.tex

\chapter{Conclusion}
\label{chapter:conclusion}

There is a lot of information concerning work in development for clinical user interfaces on images tools views, but, in fact, there is little in multimodality image and its display in breast cancer diagnosis fields.

This master project report is a first essay, to what will be the master thesis related work dissertation and state of the art \cite{borchers2012persuasion}. It describe related systems that have been designed to provide more direct support and fundament to our research. We follow at most clinical imaging tools and personal computer-based interfaces as well as a hypothetical solution of implementation with mobile interfaces where it can help us understand the right user interface solution.

In short, we analyse and rehearsed what was the first approach to the subject-matter literature on a state of the art milestone of the project to understand and to investigate the various innovations and topics made in this field of research.

So far, there have been hardly any specific studies wherein the medical interfaces are tested and evaluated for their comprehensibility and usability to users. Pretty interfaces that hide the ugly reality of underlying data do not engender clinician trust and respect. New visual cues that provide immediate user insight into assumptions and deficiencies regarding the displayed information are required. Clinicians expect and interface to keep clear and direct with easy and intuitive usability.

Some requirements for advancing innovative imaging multimodality are not just intellectual ones, but rather social, political, and educational in nature. The development of state-of-the-art of multimodal images user interface of this kind also requires multidisciplinary expertise in a variety of areas, such as human factors and ergonomics \cite{borchers2012persuasion}, perception and graphics, linguistics, psychology, pattern recognition, statistics, engineering and computer science. The multidisciplinary nature of this research across the entire spectrum.

A review of the state of the art in the field is provided showing the increasing interest of researchers in the domain and a wide range applications where these methods can be applied.

Cancer is projected to become the world's leading cause of death by 2016, with the burden of disease shifting further towards medically underserved populations in industrialised countries and the developing world.

New approaches are required across the spectrum of cancer management, in prevention, diagnosis, treatment, education and care. If developed and tested appropriately, optical imaging technologies can play an important role in several aspects, from providing objective diagnostic screening at the community healthcare level, to enabling pathology guidance in the clinical setting.

Importantly, by delivering these technical capabilities within cost-effective platforms, the impact on public health can be magnified through expanding patient access to previously unreachable healthcare systems.