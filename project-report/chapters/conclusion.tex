%!TEX root = ../dissertation.tex

\chapter{Conclusion}
\label{chapter:conclusion}

There is a lot of information concerning work in development for clinical user interfaces on images tools views, but, in fact, there is little in multimodality image and its display in breast cancer diagnosis fields.

This master project report is a first essay, to what will be the master thesis related work dissertation and state of the art [???]. It describe related systems that have been designed to provide more direct support and fundament to our research. We follow at most clinical imaging tools and personal computer-based interfaces as well as a hypothetical solution of implementation with mobile interfaces where it can help us understand the right user interface solution.

In short, we analyse and rehearsed what was the first approach to the subject-matter literature on a state of the art milestone of the project to understand and to investigate the various innovations and topics made in this field of research.

A review of the state of the art in the field is provided showing the increasing interest of researchers in the domain and a wide range applications where these methods can be applied.

Cancer is projected to become the world's leading cause of death by 2016, with the burden of disease shifting further towards medically underserved populations in industrialised countries and the developing world.

New approaches are required across the spectrum of cancer management, in prevention, diagnosis, treatment, education and care. If developed and tested appropriately, optical imaging technologies can play an important role in several aspects, from providing objective diagnostic screening at the community healthcare level, to enabling pathology guidance in the clinical setting.

Importantly, by delivering these technical capabilities within cost-effective platforms, the impact on public health can be magnified through expanding patient access to previously unreachable healthcare systems.